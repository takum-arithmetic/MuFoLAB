\documentclass[a4,oneside]{scrbook}

%%% language, encoding, font, date %%%
\usepackage[american]{babel}
\usepackage[T1]{fontenc}
\usepackage[utf8]{inputenc}
\usepackage[american]{isodate}

%%% math %%%
\usepackage{amsmath}
\usepackage{amsthm}
\usepackage{bm}

%%% graphics, plots %%%
\usepackage{graphicx}

%%% misc %%%
\usepackage{subfig}
\usepackage[margin=2cm]{geometry}

\begin{document}

\begin{figure*}[tbp]
	\begin{center}
        \subfloat[8 bits]{
			\includegraphics{eigen_graph_general/08-values}
			\includegraphics{eigen_graph_general/08-vectors}
        }\\
        \subfloat[16 bits]{
			\includegraphics{eigen_graph_general/16-values}
			\includegraphics{eigen_graph_general/16-vectors}
        }\\
        \subfloat[32 bits]{
			\includegraphics{eigen_graph_general/32-values}
			\includegraphics{eigen_graph_general/32-vectors}
        }\\
        \subfloat[64 bits]{
			\includegraphics{eigen_graph_general/64-values}
			\includegraphics{eigen_graph_general/64-vectors}
        }
	\end{center}
	\caption{
        Cumulative error distribution of the relative errors of
        the $10$ largest eigenvalues (left) and their corresponding eigenvectors (right)
        of the general matrices computed using a range of machine number types.
        The symbol $\infty_\omega$ denotes where the \textsc{Arnoldi}
        method did not converge, $\infty_\sigma$ denotes where the dynamic
        range of the matrix entries exceeded the target number type.
	}
	\label{fig:general}
\end{figure*}

\begin{figure*}[tbp]
	\begin{center}
        \subfloat[8 bits]{
			\includegraphics{eigen_graph_biological/08-values}
			\includegraphics{eigen_graph_biological/08-vectors}
        }\\
        \subfloat[16 bits]{
			\includegraphics{eigen_graph_biological/16-values}
			\includegraphics{eigen_graph_biological/16-vectors}
        }\\
        \subfloat[32 bits]{
			\includegraphics{eigen_graph_biological/32-values}
			\includegraphics{eigen_graph_biological/32-vectors}
        }\\
        \subfloat[64 bits]{
			\includegraphics{eigen_graph_biological/64-values}
			\includegraphics{eigen_graph_biological/64-vectors}
        }
	\end{center}
	\caption{
        Cumulative error distribution of the relative errors of
        the $10$ largest eigenvalues (left) and their corresponding eigenvectors (right)
        of the biological graph symmetrized, normalized Laplacian
        matrices computed using a range of machine number types.
        The symbol $\infty_\omega$ denotes where the \textsc{Arnoldi}
        method did not converge.
	}
	\label{fig:biological}
\end{figure*}

\begin{figure*}[tbp]
	\begin{center}
        \subfloat[8 bits]{
			\includegraphics{eigen_graph_infrastructure/08-values}
			\includegraphics{eigen_graph_infrastructure/08-vectors}
        }\\
        \subfloat[16 bits]{
			\includegraphics{eigen_graph_infrastructure/16-values}
			\includegraphics{eigen_graph_infrastructure/16-vectors}
        }\\
        \subfloat[32 bits]{
			\includegraphics{eigen_graph_infrastructure/32-values}
			\includegraphics{eigen_graph_infrastructure/32-vectors}
        }\\
        \subfloat[64 bits]{
			\includegraphics{eigen_graph_infrastructure/64-values}
			\includegraphics{eigen_graph_infrastructure/64-vectors}
        }
	\end{center}
	\caption{
        Cumulative error distribution of the relative errors of
        the $10$ largest eigenvalues (left) and their corresponding eigenvectors (right)
        of the infrastructure graph symmetrized, normalized Laplacian
        matrices computed using a range of machine number types.
        The symbol $\infty_\omega$ denotes where the \textsc{Arnoldi}
        method did not converge.
	}
	\label{fig:infrastructure}
\end{figure*}

\begin{figure*}[tbp]
	\begin{center}
        \subfloat[8 bits]{
			\includegraphics{eigen_graph_social/08-values}
			\includegraphics{eigen_graph_social/08-vectors}
        }\\
        \subfloat[16 bits]{
			\includegraphics{eigen_graph_social/16-values}
			\includegraphics{eigen_graph_social/16-vectors}
        }\\
        \subfloat[32 bits]{
			\includegraphics{eigen_graph_social/32-values}
			\includegraphics{eigen_graph_social/32-vectors}
        }\\
        \subfloat[64 bits]{
			\includegraphics{eigen_graph_social/64-values}
			\includegraphics{eigen_graph_social/64-vectors}
        }
	\end{center}
	\caption{
        Cumulative error distribution of the relative errors of
        the $10$ largest eigenvalues (left) and their corresponding eigenvectors (right)
        of the social graph symmetrized, normalized Laplacian
        matrices computed using a range of machine number types.
        The symbol $\infty_\omega$ denotes where the \textsc{Arnoldi}
        method did not converge.
	}
	\label{fig:social}
\end{figure*}

\begin{figure*}[tbp]
	\begin{center}
        \subfloat[8 bits]{
			\includegraphics{eigen_graph_misc/08-values}
			\includegraphics{eigen_graph_misc/08-vectors}
        }\\
        \subfloat[16 bits]{
			\includegraphics{eigen_graph_misc/16-values}
			\includegraphics{eigen_graph_misc/16-vectors}
        }\\
        \subfloat[32 bits]{
			\includegraphics{eigen_graph_misc/32-values}
			\includegraphics{eigen_graph_misc/32-vectors}
        }\\
        \subfloat[64 bits]{
			\includegraphics{eigen_graph_misc/64-values}
			\includegraphics{eigen_graph_misc/64-vectors}
        }
	\end{center}
	\caption{
        Cumulative error distribution of the relative errors of
        the $10$ largest eigenvalues (left) and their corresponding eigenvectors (right)
        of the miscellaneous graph symmetrized, normalized Laplacian
        matrices computed using a range of machine number types.
        The symbol $\infty_\omega$ denotes where the \textsc{Arnoldi}
        method did not converge, $\infty_\sigma$ denotes where the dynamic
        range of the matrix entries exceeded the target number type.
	}
	\label{fig:misc}
\end{figure*}

\end{document}
